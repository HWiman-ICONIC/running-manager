\documentstyle[a4,texhelp]{report}
\title{80/20 Running Manager v1.7}
\author{\center{
\image{}{8020Running.png}

\urlref{ReVisitor AB}{http://www.wiman.nu/rm}}}

\today
\document{
\maketitle



\tableofcontents

\copyright{} \bf{2016 (Software) \urlref{ReVisitor AB}{http://www.wiman.nu/rm}}

\copyright{} \bf{2016 (Training program) \urlref{Matt Fitzgerald}{http://www.mattfitzgerald.org}}

\chapter{80/20 Running Manager}\label{Intro}

\center{\image{}{8020Running.png}}

80/20 Running Manager helps you to create and follow your training program for 5km, 10km, half and full marathon.
It is based on the book "80/20 Running" by Matt Fitzgerald. It is strongly recommended that you buy the book
to get the most out of 80/20 Running Manager. You can buy the book from Amazon within the Running Manager program.

From the back cover:

\em{TRAIN EASIER TO RUN FASTER

This revolutionary training method has been embraced by elite runners�with extraordinary results�and now you can do it, too.

Respected running and fitness expert Matt Fitzgerald explains how the 80/20 running program�in which you do 80 percent of runs at a lower intensity and just 20 percent at a higher intensity�is the best change runners of all abilities can make to improve their performance. With a thorough examination of the science and research behind this training method, 80/20 Running is a hands-on guide for runners of all levels with training programs for 5K, 10K, half-marathon, and marathon distances.

In 80/20 Running, you�ll discover how to transform your workouts to avoid burnout.

�Runs will become more pleasant and less draining

�You�ll carry less fatigue from one run to the next

�Your performance will improve in the few high-intensity runs

�Your fitness levels will reach new heights

80/20 Running promotes a message that all runners�as well as cyclists, triathletes, and even weight-loss seekers�can embrace: Get better results by making the majority of your workouts easier.}

\center{
\begin{figure}
\center{\image{}{book.png}}

\caption{Buy the book within Running Manager.}
\end{figure}
}

\chapter{Training program}\label{TrainingProgram}
80/20 Running Manager assumes that you will train with a specific race in mind. Different training programs
are created depending on the race distance and the amount of training you are willing to do. The length of 
each training program depends on the race distance. All training programs start with a Base phase followed by 
a Peak phase and a Taper phase. They are all based on three week cycles going from easier to harder within 
each three week cycle, and starting each three week cycle at a slightly higher level than the previous.

You create a training program in the \helpref{settings window}{Settings}, but first you may want to read more
about the \helpref{layout}{Layout} of the 80/20 Running Manager.

\chapter{80/20 Running Manager Layout}\label{Layout}
When you have started 80/20 Running Manager you will see a number of panes. 

\itemize{
\item{Each pane can be dragged out of the 
main frame to a floating window by clicking the gray title bar and drag it while keeping the mouse button pressed.}
\item{You can in the same way rearrange panes or dock floatinng panes to another place within the main frame. Shaded 
hint areas will be shown as you drag the window.}
\item{The panes can be maximized by clicking the maximize button and the right side of the title bar. Click the
minimize button to restore its original position}
\item{You can close panes that you do not want to see. Closed panes can always be restored by \helpref{loading the default perspective}{perspectives}.}
\item{You can change the proportions of the docked panes by dragging the bar between two panes.}
\item{You can \helpref{save, load and delete window layouts (aka perspectives)}{perspectives}}
}

The following panes are available in Running Manager:

\itemize{
\item{ \helpref{Settings}{Settings} }
\item{ \helpref{Training table}{TrainingTable} }
\item{ \helpref{Weekday summary}{Weekday} }
\item{ \helpref{Training details}{Today} }
\item{ \helpref{Zones}{Zone} }
\item{ \helpref{Training type}{TrainingType} }
\item{ \helpref{Calculator and Predictor}{Calculators} }
\item{ \helpref{Help}{Help} }
\item{ \helpref{Link to book}{Intro} }
}

\center{
\begin{figure}
\center{\image{}{mainframe.png}}

\caption{80/20 Running Manager main frame.}
\end{figure}
}

\center{
\begin{figure}
\center{\image{}{linux.png}}

\caption{80/20 Running Manager main frame in Linux (Ubuntu).}
\end{figure}
}
\section{Customizing window layout}\label{perspectives}
You can save up to five window layouts, a.k.a. perspectives, so that you easily can restore them later.
When you have \helpref{customized all panes}{Layout} the way you want them, you select \it{Perspective->Save} in the menu or enter \it{Ctrl+P}.
You enter the name you want to give to this perspective and click Ok. To open a saved perspective, select \it{Perspective->Load} or
enter \it{Ctrl+L}. To restore the original window layout, choose \it{Default}. The last window layout will be used when RunningManager 
is reopened, either it was saved or not. If you want to delete one or more
perspectives, select \it{Perspective->Delete} (or enter Ctrl+D), check the perspectives you want to delete and click Ok.

You can show RunningManager in full screen mode by selecting Perspective->Show fullscreen or enter Ctrl+F. This will
maximize the display area for RunningManager. 

\center{
\begin{figure}
\center{\image{}{persp_menu.png}}

\caption{Manage your window layout with the \it{Perspectives} menu.}
\end{figure}
}

\center{
\begin{figure}
\center{\image{}{persp_save.png}}

\caption{ \it{Perspectives->Save} will save the current window layout. Up to five perspectives, including the default, can be saved.}
\end{figure}
}

\center{
\begin{figure}
\center{\image{}{persp_load.png}}

\caption{ \it{Perspectives->Load} will load a previously saved perspective.}
\end{figure}
}

\center{
\begin{figure}
\center{\image{}{persp_delete.png}}

\caption{ \it{Perspectives->Delete} will open this dialog where you can choose which perspective(s) to delete. The default perspective can not be deleted.}
\end{figure}
}

You can show RunningManager in full screen mode by selecting \it{Perspectives->Show fullscreen} or \it{Ctrl+F}. On small
monitors this can be useful to display as much information as possible.

\chapter{Settings for training program}\label{Settings}
\center{
\begin{figure}
\center{\image{}{settingsPane.png}}

\caption{Create a training program by filling in these data.}
\end{figure}
}
In the "Settings for training program" pane, you do the selection for creating a new training program. 
\itemize{
\item{ \bf{Distance}. Select your planned race distance}
\item{ \bf{Training level}. Select your training level. This defines the amount of training you will spend. For example,
Low will always have no training one day a week, while Medium and High will have only one day without training within 
each three week cycle. High level includes some double trainings on a single day.}
\item{ \bf{Distance unit}. Most of the training is defined in time rather than distance. The training that is defined by 
distance will use this metric unit.}
\item{ \bf{Lactate pulse}. You do not need to set this to create a training program, but it helps to define the \helpref{zones}{Zone} correctly.}
\item{ \bf{Race date}. Select the race date. Currently, the race date is always assumed to be on a Sunday. This will be 
changed in a later release. By clicking on the bar with the month you will get a list of all months within the year.
Clicking on the bar again will list years etc.}
\item{ \bf{Update}. Click the Update button to create/update your training program. This will refresh all other panes
with your desired settings.}
}

The last created training program will be loaded when you reopen RunningManager so that you easily can monitor
your day-by-day training.

\chapter{Training table}\label{TrainingTable}
\center{
\begin{figure}
\center{\image{}{trainingTable.png}}

\caption{Your training program is shown in a table. The size of the window has been collapsed in this figure.}
\end{figure}
}

The training table is updated each time you click the Update button in the \helpref{settings window}{Settings}. 
Except for the \it{Week} header lines, each line in the table represents one training. 
Each week is separated by a line, which includes the phase (Base, Peak, Taper), the distance and the time. Distance and 
time are based on the pace you have set in the \helpref{zones}{Zone} pane. For the High intensity level, there may be days with two trainings.
Note that there also is a pane with details for a \helpref{single day training}{Today}. This pane is updated whenever you 
click on a line in the training table.
The following entries exist for each training:

\itemize{
\item{ \bf{Date}. The date for the training. }
\item{ \bf{Training}. The type of training to exercise. The different types of training are defined in the \helpref{training 
type}{TrainingType} pane.}
\item{ \bf{Level}. Each \helpref{training type}{TrainingType} has a number of levels which define the length of the 
training, usually measured in time, but for long distance runs defined in distance. The explicit meaning of each level
is spelled out in the following Part and Time/Distance columns.
\item{ \bf{Part X}. Each training consists of one or more parts. Each part is defined by a specified duration in a 
\helpref{Zone}{Zone}. All parts on a line (i.e. for a training) are exercised in one sequence without any rest in between. 
Some parts include repetitions like 6x(Zone 5 Zone 1) | 6x(1minutes 2 minutes). This means that you do six repetitions
were each repetition includes 1 minute in Zone 5 and 2 minutes in Zone 1. Note that although most time is spent in Zone 1
the entire part is classed as high intensity and thus is a part of the 20% you should spend in moderate to high intensity! }
\item{ \bf{Time/distance X}. The duration of the part listed in the previous column. Each part and time/distance is colured
by the intensity in that zone.}
}

\chapter{Weekday Summary and Editor}\label{Weekday}
\center{
\begin{figure}
\center{\image{}{weekday.png}}

\caption{This pane shows the number of times you will do different training types on each weekday. You can rearrange the weekdays by
grabbing the wwkday label and drag it to a new position.}
\end{figure}
}

For each day of the week, you can see which training types you will do. You also see how many times you do each training type.
Perhaps most importantly, you can edit the weekdays. For example, long runs are by default on Sundays. If you want to do your long
runs on Saturdays, you simply drag the "Sunday" label to right after the "Friday" label. An indicator shows where your selected training
will be inserted. Click Update to update your training program.

\chapter{Training details}\label{Today}
\center{
\begin{figure}
\center{\image{}{todayPane.png}}

\caption{A list of the training of the day.}
\end{figure}
}
Initially, the \em{Training details} pane lists the training of the current date. It describes the training in detail.
The total time and distance are computed from the pace you set in the \helpref{zones}{Zone} pane. When a different date is selected by clicking a row in the
\helpref{training table}{TrainingTable}, the \em{Training details} pane is updated with information for that date.

\chapter{Zones}\label{Zone}
\center{
\begin{figure}
\center{\image{}{zones.png}}

\caption{The zones pane.}
\end{figure}

The zones pane describes the feeling and pulse you should have for each zone. To set your pulse, choose \em{File->Pulse...} in the
menu. Note that the pulse you set should be your lactate threshold pulse. One approximate way to determine your lactate threshold is 
to run as fast as you can for 30 minutes and average the pulse for the last 3 km. 

The zones are coloured green, yellow or red depending on intensity. Note that there is a gap in the pulse interval between Zone 2 and
Zone 3. This is a pulse interval you normally should not train within. Depending on which part of the training you are in (see your
\helpref{training table}{TrainingTable}) you should either go slower or faster. Also note, that the pulse is slow in reacting so
you will obvioulsy initially find your pulse in the "wrong zone" as you switch from part of a training to another. Go by feeling and
your general pace for the zone while the pulse adapts to the new part of the training.

You can optionally set a pace for each zone. This will be saved and reloaded each time RunningManager is opened. The pace is used to 
compute time and distance per week in the \helpref{training table}{TrainingTable} and per training in the \helpref{todays training}{Today}
views. The average of the min and max paces are used in these cases. 
}

\chapter{Training types}\label{TrainingType}
\center{
\begin{figure}
\center{\image{}{trainingTypesPane.png}}

\caption{The training types pane.}
\end{figure}

The training types pane briefly describes the purpose and layout of the different types of runs. They are coloured by intensity from green 
through yellow to red. All green training types are exclusively in \helpref{Zones}{Zone} 1 and 2. 

\chapter{Running calculator}\label{Calculators}
There are two tabs in this pane. One for a \helpref{calculator}{Calculator} that converts paces in different units and shows times for some
common distances. And antoher for \helpref{predicting}{Predictor} times for different distances.

\section{Calculator}\label{Calculator}
\center{
\begin{figure}
\center{\image{}{calculator.png}}

\caption{The running calculator.}
\end{figure}

Using the running calculator, you can convert between units and calculate your time for common race distances.
Change the speed with the arrows or in the text box and both the speed in the other units and the race speeds 
will be updated. The current speed is saved when closed and reloaded as you open RunningManager again.

\section{Predictor}\label{Predictor}
\center{
\begin{figure}
\center{\image{}{predictor.png}}

\caption{The running predictor.}
\end{figure}

Enter a distance and a time and the predictor will estimate which time you are likely to get for other distances.
Enter for example a time you did on a 10K race and see what times this equals for e.g. a marathon. The estimate
are obviously just approximate and assumes that you are equally well trained for the different distances.

\chapter{Use calendar in other apps.}\label{Export}
Your training program can be exported to a *.csv file by selecting \em{File->Save training calendar...} in the menu. This file can be
imported in e.g. Google calendar so that you have your training plan available wherever you are. Follow these steps to import the *.csv file
 to Google Calendar:
\itemize{
\item{Open Google Calendar on a computer.}
\item{Click on the arrow to the right of "My Calendars" and select "Create New Calendar".

\image{}{gc1.png}}
\item{Enter a Calendar Name and click "Create Calendar".

\image{}{gc2.png}}
\item{Click on the arrow to the right of "Other Calendars" and select "Import Calendar".

\image{}{gc3.png}}
\item{Browse to the *.csv file and click Open in the file menu. Select your newly created Calendar, e.g. "New York Marathon", in the
drop down list to the right of "Calendar" and click Import.

\image{}{gc4.png}}
\item{You will see a confirmation window like this.

\image{}{gc5.png}}
}
 
 Since you have put your training in a separate calendar it is easy to delete the training plan if you change your mind:
 
 \itemize{
\item{Click on the arrow that will appear to the right of the Calendar name when the cursor is over it and select "Edit calendar". }
\item{You have many options to change settings for the calendar, including a handy code snippet to include you calendar in an HTML page ("Embed this calendar").
To delete the clanedar, however, you choose "Permanently delete this calendar" under "Delete calendar". 

\image{}{gc7.png}}
\item{Confirm your choice in the next dialog. 

\image{}{gc8.png}}
 }

 \chapter{Keyboard shortcuts}\label{shortcuts}\index{Shortcuts}
 \begin{tabular}{|p{3cm}|p{12cm}|}\hline
\ruledrow{\bf{Shortcut}&\bf{Explanation}}
\row{F1&Show \helpref{help}{Help}}
\row{F8& Show About box}
\row{Ctrl+S& \helpref{Save training plan}{Export} }
\row{Ctrl+Z& Set \helpref{lactate threshold pulse}{Zone}}
\row{Ctrl+L& Load a \helpref{perspective}{perspectives}}
\row{Ctrl+P& Save a \helpref{perspective}{perspectives}}
\row{Ctrl+D& Delete \helpref{perspective(s)}{perspectives}}
\row{Ctrl+F& Toggle fullscreen mode}
\row{Alt+X& Exit the application }
\end{tabular} 

 \chapter{Help}\label{Help}
 This help is hidden when you open RunningManager. Press F1 on choose Help->Contents in the menu to show it. The help is also available
 online at \urlref{http://www.wiman.nu/runningmanager/rmhel.htm}{http://www.wiman.nu/runningmanager/rmhel.htm}. 
 
 To get your free copy of RunningManager send a mail to rm_request@wiman.nu. You will instantly get an automatic reply with
 download instructions. You can download the latest version of RunningManager 
 \urlref{here (64-bit Windows)}{http://www.wiman.nu/runningmanager/download/setup64.exe} or \urlref{here (32-bit Windows)}{http://www.wiman.nu/runningmanager/download/setup32.exe}. 
 Note that RunningManager currently only is available for the Windows platform. More platforms will be introduced soon.
 
 Press F8 or choose Help->About in the menu to show information on current software and system version.
 
 
 
 \chapter{Change log}
 \section{Version 1.1, 2015-09-02}
 \subsection{New features}
 Added \helpref{help}{Help} documentation.
 
 \section{Version 1.2, 2015-09-03}
 \subsection{New features}
 \itemize{
 \item{ Created a \urlref{homepage}{http://wiman.nu/rm}.}
 \item{ Created a 32-bit version, available at http://www.wiman.nu/download/rm_setup32.exe.}
 }
 \subsection{Bug fixes}
 \itemize{
 \item{ Upgraded to wxWidgets 3.0.2 SDK}
 \item{ Radio controls in \helpref{settings pane}{Settings} are now white, not gray.}
 \item{ Updated \helpref{help}{Help} documentation.}
 }
 
 \section{Version 1.3, 2015-09-10}
 \subsection{New features}
 \itemize{
 \item{ You can now customize the window layout by \helpref{saving and loading perspectives}{perspectives}.}
 \item{ Weeks and phases are printed out in the \helpref{training table}{TrainingTable}.}
 \item{ When selecting a date in the \helpref{training program}{TrainingTable}, the \helpref{single-day pane}{Today} is updated to display the selected day.}
 \item{ When starting RunningManager, the \helpref{training program}{TrainingTable} marks and scrolls to today's training.}
 \item{ Added tooltips to \helpref{setting}{Settings} controls.}
 }
 \subsection{Bug fixes}
 \itemize{
 \item{ Added a warning if \helpref{entered race day}{Settings} is not a Sunday (currently the only supported race day).}
 }
 
 \section{Version 1.4, 2015-10-07}
 \subsection{New features}
 \itemize{
 \item{ RunningManager can now be shown in \helpref{fullscreen mode}{perspectives}.}
 \item{ Enabled setting lactate threshold pulse in \helpref{settings}{Settings}.}
 \item{ Added quick guide to \helpref{keyboard shortcuts}{shortcuts}.}
 \item{ You can now set pace for each \helpref{zone}{Zone}.}
 }
 
 \subsection{Bug fixes}
 \itemize{
 \item{ \helpref{Single day training}{Today} was not updated on left click in \helpref{training table}{TrainingTable} in some cases. Fixed.}
 }
 
 \section{Version 1.5, 2015-12-16}
 \subsection{New features}
 \itemize{
 \item{ Time and distance are now shown per week in the \helpref{training program}{TrainingTable} and per day in the \helpref{single day training}{Today} view.}
 \item{ More information is shown in the \helpref{single day training}{Today} view.}
 }
 
 \subsection{Bug fixes}
 \itemize{
 \item{ \helpref{Single day training}{Today} did not show second training on the same day (which occurs in the high level training).}
 \item{ The automatic saving of last settings selection, pace, perspectives etc was not correctly done. Fixed, but the settings from erlier versions will need to be entered again.}
 }
 
 \section{Version 1.6, 2016-04-07}
 \subsection{New features}
 \itemize{
 \item{ Added a \helpref{calculator}{Calculator} for km/h, min/km, min/mile and for common race distances.}
 }
 
 \subsection{Bug fixes}
 \itemize{
 \item{ Fixed a bug where dates were shifted, probably due to leap year. }
 }
 
 \section{Version 1.7, 2016-10-12}
 \subsection{New features}
 \itemize{
 \item{ Added a race \helpref{predictor}{Predictor}.}
 \item{ Added a \helpref{weekday summary}{Weekday} where you can see and edit the different training types per day of week.}
 \item{ Added several default \helpref{perspectives}{perspectives}.}
 }
 
 \subsection{Bug fixes}
 \itemize{
 \item{ Fixed wrong data for Long Run Fast Finish. }
 }

}






